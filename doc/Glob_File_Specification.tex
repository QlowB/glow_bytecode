% 
% The Glow Object File Format Specification
% author: Nicolas Winkler
%
% Copyright (c)  2015  Nicolas Winkler.
% Permission is granted to copy, distribute and/or modify this document
% under the terms of the GNU Free Documentation License, Version 1.3
% or any later version published by the Free Software Foundation;
% with no Invariant Sections, no Front-Cover Texts, and no Back-Cover Texts.
% A copy of the license is distributed along this file as a separate document 
% named "GNU_Free_Documentation_License.pdf".
%



\documentclass[12pt]{article}

\usepackage{listings}
\usepackage{hyperref}

\lstnewenvironment{code}[1][]%
{
   \noindent
   \minipage{\linewidth} 
   \vspace{0.5\baselineskip}
   \lstset{basicstyle=\ttfamily\footnotesize,frame=none,#1}}
{\endminipage}


\begin{document}

\begin{titlepage}
    \begin{center}
        \vspace*{1cm}
        {
            \Huge        
            \textbf{The Glow Object File Format Specification}
        }

        \vspace{0.7cm}
        % subtitle
        
        
        \vspace{0.5cm}
        { \large
        \textbf{21 May 2015}}

        \vspace{1cm}

        { \Large
        \textbf{Nicolas Winkler}
        }
        \vfill

\bigskip
\begin{quote}
    Copyright \copyright{} 2015 Nicolas Winkler.
    Permission is granted to copy, distribute and/or modify this document
    under the terms of the GNU Free Documentation License, Version 1.3
    or any later version published by the Free Software Foundation;
    with no Invariant Sections, no Front-Cover Texts, and no Back-Cover Texts.
    A copy of the license is distributed along with this file as a separate document 
    named ``GNU\_Free\_Documentation\_License.pdf''.
\end{quote}
\bigskip

        \vspace{1cm}
        
    \end{center}
\end{titlepage}

\setlength{\parindent}{0cm}
\setlength{\parskip}{0.2cm}

\section*{The Glow Object File Format}

A Glow object file is a file containing a piece of Glow bytecode, which
can either be run directly by a Glow virtual machine, or compiled using
a Glow compiler tool.
It can be viewed as the equivalent to *.class files for Java classes or
*.obj/*.o files for C/C++ objects.



The Glow object file format uses little-endian encoding for all integers
larger than one byte. The hexadecimal 32-bit constant \texttt{0xABCDEF01} would
therefore be stored as the byte sequence \texttt{\{0x01, 0xEF, 0xCD, 0xAB\}},
which is perfectly counter-intuitive. But it's how the popular CPU architectures
nowadays work.

Glow object files constist of
\begin{itemize}
\item header
\item symbol table
\item constant table
\item bytecode
\end{itemize}

\subsection*{Header}
The header is a fixed-size data structure at the beginning of every Glow object file.

\begin{code}[language=C]
structure header {
    4 bytes: magic number;
    8 bytes: version;
    4 bytes: symbol table offset;
    4 bytes: symbol table length;
    4 bytes: constant table offset;
    4 bytes: constant table length;
    4 bytes: bytecode offset;
    4 bytes: bytecode length;
}
\end{code}

\paragraph{magic number}
    These four bytes identify the Glow object file format.
    Their value is the ASCII representation of "GLOB" which
    corresponds to the hexadecimal value \texttt{0x424F4C47}.

\paragraph{version}
    This entry specifies the version of the object file. The two most
    significant bytes (note: these are the last and the second last of the 8 bytes,
    say thanks to little-endian) stand for the main version, where the two following
    bytes can be increased on little changes. The remaining 4 bytes may be used in the
    future, or can be used for very very small changes if needed.

\paragraph{symbol table offset}
    This value contains an offset pointing from the beginning of the file to the first entry
    of the symbol table, which for one thing holds references to the methods and classes in the
    current object file and for another thing also contains symbol names pointing to
    extern Glow object files.

\paragraph{symbol table length}
    Specifies the number of entries in the symbol table. Since all entries in this table
    are the same size, the whole size of this table can be calculated from this value.

\paragraph{constant table offset}
    This value contains an offset pointing from the beginning of the file to the first entry
    of the constant table, a table, which stores constants, which are too big to store
    directly in the bytecode (e.g strings). 

\paragraph{constant table length}
    Specifies the number of entries in the constant pool table. Note that the entries of this
    table may vary in size, so the absolute size of the table can't yet be decided.

\paragraph{bytecode offset}
    Defines the offset from the beginning of the file to the start of the bytecode sequence.

\paragraph{bytecode length}
    This value contains the number of bytes the bytecode part of the file takes up.

\subsection*{Symbol Table}
This section constists of a table with entries for each class used and for each function called in the bytecode.

Each entry is of the following format:

\begin{code}[language=C]
structure symbol_table_entry {
    1 byte:  type;
    4 bytes: argument;
}
\end{code}

\paragraph{type} is always one of the following values:
 
\begin{tabular}{| l | l |}
\hline
\verb|symbol_table_entry::type| & Value \\
\hline \hline
\verb|GLOB_CLASS_REFERENCE| & 0x01 \\
\hline
\verb|GLOW_METHOD_REFERENCE| & 0x02 \\
\hline
\verb|GLOB_CLASS_DEFINITION| & 0x03 \\
\hline
\verb|GLOB_METHOD_DEFINITION| & 0x04 \\
\hline
\end{tabular}

\paragraph{argument}
    The argument is a reference to the constant table entry containing the class/method name.

Several bytecode instructions will contain a link to an entry of this table:

\begin{itemize}
\item \verb|GLOW_ALLOCATE_OBJECT = 0xB0| is followed by an index pointing to an entry in this
    table. This entry must therefore contain information about the class of which an instance
    should be allocated.
\item \verb|GLOW_CALL_|... like instructions are also followed by an index pointing to a function
    entry in this table.
\end{itemize}

\subsection*{Constant Table}

This table stores big constant values like strings or class/method names.

Entries are of the following format:

\begin{code}[language=C]
structure constant_table_entry {
    1 byte:       type;
    4 bytes:      length;
    length bytes: data;
}
\end{code}

\paragraph{type} identifies the type of the constant. This can be one of the following values:

\begin{tabular}{| l | l |}
\hline
\verb|constant_table_entry::type| & Value \\
\hline \hline
\verb|GLOB_CONSTANT_STRING| & 0x01 \\
\hline
\end{tabular}

\paragraph{length} specifies the size (in bytes) of the data part of the entry.

\paragraph{data} is the actual data in the entry.

The entries follow each other directly i.e. there is no space between two entries. The offset
to the next entry can therefore be computed using the formula: \(5 + \verb|length|\).


\subsection*{Bytecode}

This is the simplest part of the file. It just contains the plain bytecode content, ready for interpretation
by an interpreter or further compilation to architecture-specific machine code.

For more information about the Glow bytecode, please read the Glow Bytecode Specification.

\end{document}



