
\title{The Glow Object File Specification}
\author{Nicolas Winkler}
\date{20 May 2015}

\documentclass[12pt]{article}

\begin{document}
\maketitle

\section{General}

A Glow Object File is a file containing a piece of Glow bytecode, which
can either be run directly by a Glow virtual machine, or compiled using
a Glow compiler tool.
It can be viewed as the equivalent to *.class files for Java classes or
*.obj/*.o files for C/C++ objects.



\section{Header}
The Glow Object File is uses little-endian encoding for all integers
larger than one byte. The hexadecimal 32-bit constant \texttt{0xABCDEF01} would
therefore be stored as the byte sequence \texttt{\{0x01, 0xEF, 0xCD, 0xAB\}},
which is perfectly counter-intuitive. But it's how the popular CPU architectures
nowadays work.



\end{document}
